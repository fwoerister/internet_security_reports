\section*{rep-10}
The code in figure \ref{fig:rep10} generates the plot for the first subtask of this exercise. It generates a plot, showing the \textit{\#pkts/hour}. The only difference to the other subtasks is in line 6 of this code snippet. Here the parameter 3 describes the column index of the data we want to plot. The indices for subtask 2-4 can be seen in table \ref{tab:indices}.
\begin{table}[H]
\center
\begin{tabular}{lr}
\toprule
index & feature name \\
\midrule
2 & \#bytes / hour (daily avg.) \\
4 & \#uIPs / hour (daily avg.) \\
5 & \#uIPd / hour (daily avg.) \\
\bottomrule
\end{tabular}
\caption{Indices for the several dataset features}
\label{tab:indices}
\end{table}
\begin{figure}[H]
\lstinputlisting{./chapters/matlab/team16_ex3_rep10_1.m}
\caption{Matlab code for first subtask of rep-10}
\label{fig:rep10}
\end{figure}

In addition a function was implemented, for plotting a graph, which also deals with the epoch to datenum format conversion. The exact implementation of this function is listed in figure \ref{fig:plotDarkNetData}.

\begin{figure}[H]
\lstinputlisting{./chapters/matlab/plotDarknetData.m}
\caption{Matlab code for the 'plotDarknetData' function}
\label{fig:plotDarkNetData}
\end{figure}

For the normalization we simply loaded every feature set in a separate feature vector and devided it by the vectors max value. In figure \ref{fig:normalization} the normalization is demonstrated on the first feature \(\#pkts/hour\). The smoothing, as described in the exercise description, was implemented in the matlab function 'smoothLine'.

\begin{figure}[H]
\lstinputlisting{./chapters/matlab/team16_ex3_rep10_5_normalization.m}
\caption{Matlab code for the normalization}
\label{fig:normalization}
\end{figure}


\subsection*{Results}
\begin{figure}[H]
\center
\includegraphics[width=.7\textwidth]{./chapters/plots/rep10_1.jpg}\\
\caption{Results of rep10-1}
\end{figure}
\begin{figure}[H]
\center
\includegraphics[width=.7\textwidth]{./chapters/plots/rep10_2.jpg}\\
\caption{Results of rep10-2}
\end{figure}
\begin{figure}[H]
\center
\includegraphics[width=.7\textwidth]{./chapters/plots/rep10_3.jpg}\\
\caption{Results of rep10-3}
\end{figure}
\begin{figure}[H]
\center
\includegraphics[width=.7\textwidth]{./chapters/plots/rep10_4.jpg}\\
\caption{Results of rep10-4}
\end{figure}

\section*{rep-11}
For this task, the correlation matrix was calculated by executing the matlab code of figure \ref{fig:correlation}
\begin{figure}[H]
\lstinputlisting{./chapters/matlab/team16_ex3_rep11.m}
\caption{Matlab code for the normalization}
\label{fig:correlation}
\end{figure}
The resulting correlation matrix (figure \ref{matrix:correlation}) shows us, that the \text{\#unique source IP addresses per hour} show the lowest correlation values, especially with the \text{\#unique destination IP addresses per hour}


\section*{rep-12}

\section*{rep-13}

\section*{rep-14}

\section*{rep-15}

\section*{rep-16}

\section*{rep-17}

\section*{rep-18}

\section*{rep-19}

\section*{rep-20}

\section*{rep-21}

\section*{rep-22}

\section*{rep-23}

