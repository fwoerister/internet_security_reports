\section*{Exercise 3}

\subsection*{rep-10}
The code in listing~\ref{lst:rep10} generates the plot for the first subtask of this exercise, showing the \textit{\#pkts/hour}. The only difference to the other subtasks is in line 6 of this code snippet. Here the parameter 3 describes the column index of the data we want to plot. The indices for subtask 2-4 can be seen in table \ref{tab:indices}.  \TODO{maybe kick the implementation part}
\begin{table}[H]
\center
\begin{tabular}{lr}
\toprule
index & feature name \\
\midrule
2 & \#bytes / hour (daily avg.) \\
4 & \#uIPs / hour (daily avg.) \\
5 & \#uIPd / hour (daily avg.) \\
\bottomrule
\end{tabular}
\caption{Indices of the dataset features}
\label{tab:indices}
\end{table}

\lstinputlisting[label={lst:rep10},
				caption=Code for first subtask of rep-10]
				{./chapters/matlab/team16_ex3_rep10_1.m}
				

In addition a function was implemented, for plotting a graph, which also deals with the epoch to datenum format conversion. The exact implementation of this function is listed in listing~\ref{lst:plotDarkNetData}.
 \TODO{i would kick that too}

\lstinputlisting[label={lst:plotDarkNetData},
				caption=Code for the 'plotDarknetData' function]
				{./chapters/matlab/plotDarknetData.m}

Subtask 5 requires to normalize the data shown in Subtask 1 - 4. For the normalization we simply loaded every feature set in a separate feature vector and devided it by the vectors max value. In listing~\ref{lst:normalization} the normalization is demonstrated for the first feature \(\#pkts/hour\). The smoothing was implemented in the function 'smoothLine'. It was implemented as described in the exercise description.

\lstinputlisting[label={lst:normalization},
				caption=Code for the normalization]
				{./chapters/matlab/team16_ex3_rep10_5_normalization.m}

\subsubsection*{Results}
In the next figures the required plots are presented.
\begin{figure}[H]
\center
\includegraphics[width=.7\textwidth]{./chapters/plots/rep10_1.jpg}\\
\caption{Results of rep10-1}
\end{figure}
\begin{figure}[H]

\center
\includegraphics[width=.7\textwidth]{./chapters/plots/rep10_2.jpg}\\
\caption{Results of rep10-2}
\end{figure}

\begin{figure}[H]
\center
\includegraphics[width=.7\textwidth]{./chapters/plots/rep10_3.jpg}\\
\caption{Results of rep10-3}
\end{figure}

\begin{figure}[H]
\center
\includegraphics[width=.7\textwidth]{./chapters/plots/rep10_4.jpg}\\
\caption{Results of rep10-4}
\end{figure}

\begin{figure}[H]
\center
\includegraphics[width=.7\textwidth]{./chapters/plots/rep10_5.jpg}\\
\caption{Results of rep10-5}
\end{figure}





\subsection*{rep-11}
For this task, the correlation matrix was calculated by executing the matlab code of listing~\ref{lst:correlation}.

\lstinputlisting[label={lst:correlation},
				caption=Matlab code for calculating the minimal correlation]
				{./chapters/matlab/team16_ex3_rep11.m}

The correlation matrix presented in figure~\ref{fig:correlation} shows us, that the \text{\#unique source IP addresses per hour} show the lowest correlation values, especially with the \textit{\#unique destination IP addresses per hour}. It is noticeable, that there is a drop after Jan/16 for the unique IP sources. This drop does slightly correlate with the other features. This may be a result of a botnet stopping connecting to the darkspace. The botnet may tried to do a horizontal scan.

\begin{figure}[H]
\center
$
\begin{bmatrix}
1.000 &  0.965 &  0.720 &  0.934 \\
0.965 &  1.000 &  0.610 &  0.973 \\
0.720 &  0.610 &  1.000 &  0.589 \\
0.934 &  0.973 &  0.589 &  1.000
\end{bmatrix}
$
\caption{ Correlation matrix of the dataset containing the values for all 4 features }
\label{fig:correlation}
\end{figure}

\subsection*{rep-12}
In order to find out, if there are more sources sending data to the darkspace, than addresses receiving data from it, the ratio between the unique source and destination IPs is calculated, as shown in listing~\ref{lst:ratio_source_destination_ip}.

\lstinputlisting[label={lst:ratio_source_destination_ip},
				caption=Matlab code for calculating the ratio between the average unique source addresses and the average unique destination addresses per hour]
				{./chapters/matlab/team16_ex3_rep12_less.m}


The ratio between source and destination IPs is 9.31. This means, that there are approximately 10 times more addresses receiving data from the darkspace. This result was expected, as \textit{rep-11} already showed, that there may be a horizontal scan going on.

\subsection*{rep-13}
To find the peak of the '\#uIPs/hour (daily avg.)', the first column was converted to datenum format before retrieving the maximum value of the '\#uIPs/hour (daily avg.)' column. The \textit{max} function retrieves the maximum value and it's index. This index we use to retrieve the Date of this peak. This three steps are listed in listing~\ref{lst:find_peak}.

\TODO{maybe kick that part with the timestamps}

\lstinputlisting[label={lst:find_peak},
				caption=Matlab snipped for finding the peak of '\#uIPs/hour (daily avg.)']
				{./chapters/matlab/team16_ex3_rep13_less.m}


The peak occurred on the 15th of December 2015. As the neighbor values (14th and 16th of December) are significantly smaller, the peak lasted only for one day.

\subsection*{rep-14}
Table~\ref{tab:stat_jun2017_gen} shows the statistical values for the file \textit{Jun2017\_gen.csv} and table~\ref{tab:stat_jun2017_global} shows statistical values for the same time period, but the data was extracted from the \textit{global\_last10years.csv}. 

\begin{table}[H]
\center
\begin{tabular}{lrrrr}
\toprule
feature & total sum & mean & median & standard deviation \\
\midrule
\#pkts/hour & 6 922.229  &   9.695  &    9.937  &   1.443 \\
\#bytes/hour & 207.103  &   0.290   &   0.241  &   0.098 \\
\#uIPs/hour  & 13 157.867  &  18.428  &   18.511 &    4.329 \\
\#uIPd/hour  & 719 606.146 &  1 007.852 &   993.364 &  237.632 \\
\bottomrule
\end{tabular}
\caption{Statistics of June 2017 from Jun2017\_gen.csv (in millions)}
\label{tab:stat_jun2017_gen}
\end{table}

\begin{table}[H]
\center
\begin{tabular}{lrrrr}
\toprule
feature & total sum & mean & median & standard deviation \\
\midrule
\#pkts/hour  &    291.275  &      9.709  &      9.996   &    1.169 \\
\#bytes/hour &      8.700  &      0.290  &      0.254   &    0.076 \\
\#uIPs/hour  & 30 268.997  &  1 008.967  &  1 008.526   &  101.158\\
\#uIPd/hour  &    553.823  &     18.461  &     19.124   &    3.080 \\
\bottomrule
\end{tabular}
\caption{Statistics of June 2017 from global\_last10years.csv (in millions)}
\label{tab:stat_jun2017_global}
\end{table}

\subsection*{rep-15}
Comparing table~\ref{tab:stat_jun2017_gen} and table~\ref{tab:stat_jun2017_global} leads to the conclusion, that some values do coincide, but not all. Especially int the \textit{total sum} column are big differences. The problem is, that in the table~\ref{tab:stat_jun2017_gen} the averages for every hour are summed up and compared to the averaged value of the second table, therefore it is not surprising, that the \textit{total sum} values of~table \ref{tab:stat_jun2017_global} are significantly smaller.
There are also differences, when it comes to '\#uIPs/hour' and '\#uIPd/hour'. The problem here is, that in the table~\ref{tab:stat_jun2017_gen}, the same IP occurring every hour of the same day, is counted as a unique IP, where as in table \ref{tab:stat_jun2017_global} this IP address would be counted for the whole day as \textbf{one} unique IP address.
Mean, median and standard deviation values for '\#pkts/hour' and '\#bytes/hour' are not sensitive to those problems, therefore they do coincide in both tables.

\subsection*{rep-16}
The protocols found in the 'Jun2017\_proto.csv' file are listed in table~\ref{tab:proto}. The protocol name for each number can be found here: \url{https://www.iana.org/assignments/protocol-numbers/protocol-numbers.xhtml}

\begin{table}[H]
\center
\begin{tabular}{lrp{5cm}}
\toprule
	Protocol No. & Name & Description \\
\midrule
	1 & ICMP & The Internet Control Message Protocol is used for reporting status informations or error messages in IP, TCP and UDP protocols. Especially Gateways and Hosts use this Protocol for reporting problems. \tablefootnote{\url{https://www.itwissen.info/ICMP-Internet-control-message-protocol-ICMP-Protokoll.html}}\\
	6 & TCP & The Transmission Control Protocol is one of the main protocols of the internet. It provides reliable, ordered and error-checked delivery of bytes. \tablefootnote{\url{https://en.wikipedia.org/wiki/Transmission_Control_Protocol}} \\
	17 & UDP & The User Datagram Protocol is one of the core members of the internet. The big difference to TCP is, that it is connection less, which means, that data can be send, without the need to set up communication channels or data paths. \tablefootnote{\url{https://en.wikipedia.org/wiki/User_Datagram_Protocol
}}\\
\bottomrule
\end{tabular}
\caption{ The protocols been found in 'Jun2017\_proto.csv' }
\label{tab:proto}
\end{table}

\subsection*{rep-17}
\subsubsection*{Statistics}
\begin{table}[H]
\center
\begin{tabular}{lrrr}
\toprule
protocol & mean (in millions) & median (in millions) & standard deviations \\
\midrule
ICMP &   16.525  &  16.325  &  25.355 \% \\
TCP &     1.747  &   1.674  &  39.693 \% \\
UDP &     0.184  &   0.149  &  83.086 \% \\
others &  0.019  &   0.019  &  33.273 \% \\
\bottomrule
\end{tabular}
\caption{ Packets statistics }
\label{tab:proto-stats-packets}
\end{table}

\begin{table}[H]
\center
\begin{tabular}{lrrr}
\toprule
protocol & mean (in millions) & median (in millions) & standard deviations \\
\midrule
ICMP &    0.223 & 19.105 &   29.895 \% \\
TCP &     0.077 &  0.050 &   59.264 \% \\
UDP &     0.012 &  0.004 &  276.280 \% \\
others & -0.021 & -0.005 & -123.326 \% \\
\bottomrule
\end{tabular}
\caption{ uIPs statistics }
\label{tab:proto-stats-uIPs}
\end{table}

\begin{table}[H]
\center
\begin{tabular}{lrrr}
\toprule
protocol & mean (in millions) & median (in millions) & standard deviations \\
\midrule
ICMP &    9.253 &  9.448 &  14.728 \%  \\
TCP &     1.055 &  1.049 &  35.417 \%  \\
UDP &     0.155 &  0.120 &  92.408 \%  \\
others & -0.744 & -0.724 & -40.126 \%  \\
\bottomrule
\end{tabular}
\caption{ uIPd statistics }
\label{tab:proto-stats-uIPd}
\end{table}

\subsubsection*{Boxplots}
\begin{figure}[H]
\center
\includegraphics[width=.7\textwidth]{./chapters/plots/rep17.jpg}\\
\caption{Boxplots of \textit{rep-17}}
\end{figure}


\subsection*{rep-18}
Yes, there are negative values for '\#uIPs/hour' and '\#uIPd/hour'. In case one and the same host uses all three protocols in the same hour, this ip address is going to be counted three times (for each protocol). But in the 'Jun2017\_gen.csv' file it will be counted only one time. In this case we would end up with a 'others' value in the '\#uIPs/hour' table of $1 - (1+1+1) = -2$. The same applies to the '\#uIPd/hour'. The 'packets' values are not affected, as every packet refers to \textbf{exactly one} protocol.
% \section*{rep-19} OPTIONAL
% \section*{rep-20} OPTIONAL

\subsection*{rep-21a}
\begin{figure}[H]
\center
\includegraphics[width=.7\textwidth]{./chapters/plots/rep21a.jpg}\\
\caption{}
\end{figure}

\subsection*{rep-21b}
\begin{figure}[H]
\center
\includegraphics[width=.7\textwidth]{./chapters/plots/rep21b.jpg}\\
\caption{}
\end{figure}

\subsection*{rep-21c}
\textbf{TODO:} find the exact 'k'- value of other peaks $\Rightarrow$ \textit{Additionally, comment on other possible periodicity and patterns in case that your plots show more than one periodical pattern!}
\subsection*{rep-22}
\begin{figure}[H]
\center
\includegraphics[width=1\textwidth]{./chapters/plots/rep-22.jpg}\\
\caption{}
\end{figure}

\subsection*{rep-23a}
\textbf{TODO}
\subsection*{rep-23b}
\textbf{TODO}
\subsection*{rep-23c}
\textbf{TODO}
\subsection*{rep-23d}
\textbf{TODO}
