\section*{Exercise 2.1a - Pcap format}
\subsection*{rep-4: First three rows of pcap file}
The lines have been obtained with the following command:
\begin{verbatim}
cat synscan.pcap | tcpdump -t -nr - | head -3 > rep-4
\end{verbatim}
The lines are:
\begin{verbatim}
IP 192.168.83.24.848 > 192.168.83.1.2049: Flags [P.], seq 3466753733:3466753893, ack 4109170401, 
win 3012, options [nop,nop,TS val 407079 ecr 950719840], 
length 160: NFS request xid 1469516420 156 getattr fh 0,0/22

IP 192.168.83.1.2049 > 192.168.83.24.848: Flags [P.], seq 1:129, ack 160, win 5999, 
options [nop,nop,TS val 950719851 ecr 407079], 
length 128: NFS reply xid 1469516420 reply ok 124 getattr NON 3 ids 0/3 sz 0

IP 192.168.83.24.848 > 192.168.83.1.2049: Flags [.], ack 129, win 3012, 
options [nop,nop,TS val 407079 ecr 950719851], length 0
\end{verbatim}

\section*{Exercise 2.1b - Flow Record format}
\subsection*{rep-5: First five rows of flow record}
With the following command the pcap file has been converted:
\begin{verbatim}
rwptoflow --flow-output=dksp_flow_comp.rw synscan.pcap --compression-method=zlib
\end{verbatim}
Next command returns the first five lines and the header:
\begin{verbatim}
rwcut dksp_flow_comp.rw | head -6 > rep-5
\end{verbatim}
The lines are:
\begin{verbatim}
          sIP|           dIP|sPort|dPort|pro|packets|bytes|flags|                  sTime|
          duration|                  eTime|ttl|
          
192.168.83.24|  192.168.83.1|  848| 2049|  6|      1|  212|PA   |2018/05/09T11:23:11.391|
             0.000|2018/05/09T11:23:11.391| 64|
             
 192.168.83.1| 192.168.83.24| 2049|  848|  6|      1|  180|PA   |2018/05/09T11:23:11.391|
             0.000|2018/05/09T11:23:11.391| 64|
             
192.168.83.24|  192.168.83.1|  848| 2049|  6|      1|   52| A   |2018/05/09T11:23:11.391|
             0.000|2018/05/09T11:23:11.391| 64|
             
192.168.83.24|  192.168.83.1|  848| 2049|  6|      1|  288|PA   |2018/05/09T11:23:11.391|
             0.000|2018/05/09T11:23:11.391| 64|
             
 192.168.83.1| 192.168.83.24| 2049|  848|  6|      1|  108|PA   |2018/05/09T11:23:11.391|
             0.000|2018/05/09T11:23:11.391| 64|
\end{verbatim}

\section*{Exercise 2.2 - Extracting data rates}

\subsection*{rep-6 Time interval values of the data file}
\subsubsection*{Start time}
The human readable start time has been obtained with the following command:
\begin{verbatim}
rwuniq --sort-output --delimited=',' --no-title --fields=stime
  ~/workfiles/team16.flowrecord.rw | head -1
\end{verbatim}
The output is \texttt{2012/04/26T00:00:00,26482}.

The epoch version has been obtained with the following command:
\begin{verbatim}
rwuniq --sort-output --delimited=',' --no-title --fields=stime
  --epoch-time ~/workfiles/team16.flowrecord.rw | head -1
\end{verbatim}
The output is \texttt{1335398400,2648}.

\subsubsection*{End time}
The human readable start time has been obtained with the following command:
\begin{verbatim}
rwuniq --sort-output --delimited=',' --no-title --fields=stime
  ~/workfiles/team16.flowrecord.rw | tail -1
\end{verbatim}
The output is \texttt{2012/04/26T00:59:59,26775}.

The epoch version has been obtained with the following command:
\begin{verbatim}
rwuniq --sort-output --delimited=',' --no-title --fields=stime
  --epoch-time ~/workfiles/team16.flowrecord.rw | tail -1
\end{verbatim}
The output is \texttt{1335401999,267751}.


\subsection*{rep-7 Number of packets per hour}
The number of packets have been obtained with the following command:
\begin{verbatim}
rwuniq --sort-output --delimited=',' --no-title --fields=stime
  --value=packet --bin-time=3600 ~/workfiles/team16.flowrecord.rw
\end{verbatim}
The result is \texttt{2012/04/26T00:00:00,127348359}

\subsection*{rep-8 Number of unique IP source addresses}
The number of unique IP source addresses have been obtained with the following command:
\begin{verbatim}
rwuniq --sort-output --delimited=',' --no-title --fields=stime
  --value=sIP-Distinct --bin-time=3600 ~/workfiles/team16.flowrecord.rw
\end{verbatim}
The output is \texttt{2012/04/26T00:00:00,1620232}.

\subsection*{rep-9}
The darkspace contains 1/256th of all available IP addresses. Therefore we can estimate, that the pollution the entire Internet is the 256 times the pollution of the darkspace. Realistically the real value is bigger, because the darkspace addresses are known and therefore the pollution may not be directed there. Attackers don't  any benefit of directing their traffic there. We think the Internet is polluted. The pollution causes a higher bandwidth demand across the whole Internet. Additionally extra computing power is used to process this traffic.
